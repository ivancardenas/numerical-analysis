\documentclass{article}

\usepackage{algorithm}
\usepackage{algpseudocode}

\begin{document}

  \begin{algorithm}
    \caption{Newton's Method}
    \begin{algorithmic}[1]
      \Procedure{Newton}{$x_{0},\ tol,\ iter$}
        \State $y \gets f(x_{0})$
        \State $dy \gets f'(x_{0})$
        \State $count \gets 0$
        \State $E \gets tol + 1$
        \While {$y \not= 0$ \textbf{ and } $dy \not= 0$ \textbf{ and } $E > tol$ \textbf{ and } $count < iter$}
          \State $x_{1} \gets x_{0} - (y/dy)$
          \State $y \gets f(x_{1})$
          \State $dy \gets f'(x_{1})$
          \State $E \gets abs(x_{1} - x_{0})$
          \State $x_{0} \gets x_{1}$
          \State $count \gets count + 1$
        \EndWhile
        \If {$y = 0$}
          \State \textbf{print} "$x_{0}$ is a root"
        \ElsIf {$dy = 0$}
          \State \textbf{print} "$x_{1}$ could be a multiple root"
        \ElsIf {$E < tol$}
          \State \textbf{print} "$x_{1}$ is approximated by $E < tol$"
        \Else
          \State \textbf{print} "Failure"
        \EndIf
      \EndProcedure
    \end{algorithmic}
  \end{algorithm}

\end{document}
